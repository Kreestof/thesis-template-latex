\pagenumbering{roman}
\setcounter{page}{1}

\selecthungarian

%----------------------------------------------------------------------------
% Abstract in Hungarian
%----------------------------------------------------------------------------
\chapter*{Kivonat}\addcontentsline{toc}{chapter}{Kivonat}

Modell-vezérelt fejlesztési folyamatoknál a jól-formáltság ellenőrzése rendkívül fontos. A modellre meghatározott kényszerek validációja különböző eszközökön végzett lekérdezések segítségével történik. Emellett a fejlesztési folyamat során a modellbe hibák injektálódnak és javítódnak. Az eszközöket össze lehet hasonlítani az alapján, hogy ezeket a műveleteket mennyi idő alatt végzik el. Egy ilyen teljesítménymérést végez a Train Benchmark projekt egy vasúti elemekből álló modellen. A dolgozat írásakor a Train Benchmark már elkészült tíz eszközre. Jelen szakdolgozat ismerteti a Train Benchmark működését, majd két gráfadatbázis-kezelő eszközön elvégzett mérés eredményét rögzíti. Továbbá a Graph Engine nevű Microsoft által fejlesztett eszköz működését bemutatja, illetve az erre implementált teljesítménymérést kiértékeli és összehasonlítja a két rögzített eredménnyel. Emellett egy általános képet ad a gráfalapú adatbázisok sajátosságairól. Végül összefoglalja a Graph Engine előnyeit és hátrányait.

\vfill
\selectenglish


%----------------------------------------------------------------------------
% Abstract in English
%----------------------------------------------------------------------------
\chapter*{Abstract}\addcontentsline{toc}{chapter}{Abstract}

Verifying the well-formedness of a model is highly important in model-driven developments. The validation of predetermined constraints is run by querying with different tools. In addition, faults get injected and corrected into the model during the development process. The tools can be compared based on how much time they need to be finished. The Train Benchmark project concentrates such an evaluation on model which consists of railway elements. The Train Benchmark has already been implemented for ten tools. This thesis is about to describe how the Train Benchmark operates. It also shows the performing process of the benchmarking on two graph database tools. Moreover, it demonstrates an additional tool, named Graph Engine, which was developed by Microsoft. Afterwards, it assesses the benchmarking of the Graph Engine and compares it with the results of the other two tools. Besides, it gives a general summary of the particularity of graph databases. In the end, it sums up the advantages and disadvantages of the Graph Engine.

\vfill
\selectthesislanguage

\newcounter{romanPage}
\setcounter{romanPage}{\value{page}}
\stepcounter{romanPage}