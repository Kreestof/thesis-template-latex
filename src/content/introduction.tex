%----------------------------------------------------------------------------
\chapter{\bevezetes}
%----------------------------------------------------------------------------

Mérnöki fejlesztési folyamatok során a teljesítmény kulcsfontosságú tényező. Amennyiben a mérnök megfelelő eszközzel dolgozik, a munka is hamarabb készül el. Ugyanígy van ez a számítógépen történő fejlesztések során is. Ilyen fejlesztés során az adatkezelő szoftvereket tekintjük egyfajta eszköznek, mellyel a programozó műveleteket végez a tárolt adatokon. Ezeknek az adatoknak a reprezentálására gyakran gráf modelleken történik. Felmerül a kérdés, hogy az eszközök hatékonyságát sorba lehet-e rendezni valamilyen objektív mérce alapján. A problémák sokrétűsége azt mutatja, hogy a felhasználási terület általában nagymértékben befolyásolja az eredményt. Ezek az eszközök sok szempontból különbözők, hatékonyságuk más és más problémákra rendkívül változatos. Észszerűen ezért a legtöbb teljesítménymérés egy bizonyos fajta problémára összpontosul.

Egy ilyen feladat lehet a modell-validálás, melynek célja, hogy adott modellen (melyet az adatbázisban tárolunk) megkövetelt kényszereket ellenőrizzen. Ez rendkívül fontos egy fejlesztési folyamat során, mivel visszajelzést ad a programozónak és/vagy a felhasználónak a modell helyességéről. A Train Benchmark keretrendszer célja egy ilyen folyamat szimulálása több adatbázis-kezelő szoftveren és a teljesítményük mérése, annak kiértékelése.

%Modell-vezérelt fejlesztés folyamán jól-formáltsági kényszerek ellenőrzése olyan feladat, mely végigkíséri a teljes folyamatot.

%A Train Benchmark egy projekt melynek célja ezen eszközök teljesítményalapú összehasonlítása egy modell-vezérelt fejlesztési folyamat szimulációja által.

A Train Benchmark több eszközre már meg lett valósítva.\cite{Szárnyas2017} Ez azt jelenti, hogy a mérés el lett végezve és az eszközök teljesítménye össze lett hasonlítva futásidő, memória- és tárhelyhasználat alapján.  Jelen dolgozat célja egy modern adatbázis-kezelő teljesítménymérésének implementálása, a Train Benchmark két adatbázis-rendszerre megvalósított tesztjének futtatása Windows operációs rendszeren és a kapott eredmények összehasonlítása. Ezáltal kiemelve a választott eszköz előnyeit és hátrányait a másik kettőhöz képest. A kiválasztott adatbáziskezelő-rendszer a Microsoft által fejlesztett Graph Engine.

%Egy modell-vezérelt fejlesztés leegyszerűsítve három részből áll.
%\begin{enumerate}
%	\item Jól-formáltsági kényszerek ellenőrzése a modellen. Ennek 
%\end{enumerate}


%Számos adatbázis-kezelő rendszer van a világon. A teljesítmény kulcsfontosságú jelentőséggel bír, főleg, amikor nagyméretű adatbázisokról van szó. Emiatt felmerül a kérdés, hogy sorba lehet-e őket rendezni valamilyen objektív mérce alapján. A problémák sokrétűsége azt mutatja, hogy a felhasználási terület általában nagymértékben befolyásolja az eredményt. Emellett az adatbázis-kezelő rendszereket sokféleképp kategóriába is lehet sorolni, például a matematikai modell vagy a lekérdezés nyelve alapján. Az egy kategóriába eső rendszereket általában könnyebb összehasonlítani, mint olyanokat, amik külön kategóriába esnek. Ezek a problémák szinte lehetetlen feladattá tesznek egy általános összehasonlítást. Észszerűen ezért a legtöbb teljesítménymérés egy bizonyos fajta problémára összpontosul. Egy ilyen feladat lehet a modell-validálás, melynek célja, hogy adott modellen (melyet az adatbázisban tárolunk) megkövetelt kényszereket ellenőrizzen. Ez rendkívül fontos egy fejlesztési folyamat során, mivel visszajelzést ad a programozónak és/vagy a felhasználónak a modell helyességéről. A Train Benchmark célja egy ilyen folyamat szimulálása több adatbázisrendszeren és a teljesítményük mérése, annak kiértékelése.
%
%%meg lett valósítva nem túl szép
%A Train Benchmark több adatbázis-kezelő rendszerre már meg lett valósítva. Jelen dolgozat célja egy modern adatbázis-kezelő tesztelésének implementálása, a Train Benchmark két adatbázisrendszer-tesztjének futtatása Windows operációs rendszeren és a kapott eredmények összehasonlítása. A kiválasztott adatbáziskezelő-rendszer a Microsoft által fejlesztett Graph Engine.
%
%\section{A dolgozat felépítése}
%
%Az elkészült dolgozat a következő részekből épül fel:
%
%A 2. fejezetben a Train Benchmark működésének elvét mutatjuk be. Kezdve a Train Benchmarkban használt modell elemeivel, közöttük lévő kapcsolattal, majd a megfogalmazott kényszerek részletezésével. Ezután a fejlesztési folyamat forgatókönyve, végül pedig azon környezetek (Jena, RDF4J) rövid bemutatása következik, melyeknek tesztjét lefuttattuk.
%
%A 3. fejezet az adatbázisrendszerek egy típusáról, a gráfadatbázisokról szól. Egy egyszerű példán keresztül megmutatjuk, hogy adott esetben miért lehet gyorsabb és jobb a relációs adatbázisrendszereknél.
%
%A 4. fejezet a Graph Engine nevű gráfadatbázis-rendszer felépítését és működését taglalja. A 3. fejezetben található példán keresztül mutatjuk be, hogy milyen módon kell használni.
%
%Az 5. fejezetben az általunk elkészített tesztelőprogram implementációját részletezzük. Rövid forráskódokat kiragadva általánosan bemutatjuk, hogy hogyan épül fel. Kiegészítve a Train Benchmarkhoz már elkészült tesztek Windows operációs rendszeren való futtatási problémáinak megoldásával, illetve a fejlesztés során felmerülő nehézségekre való válaszok okainak indoklásával.
%
%%TODO: utolsó két fejezethez tartozó bevezető
%
%A 6. fejezet a tesztek futtatása után kapott eredmények kiértékelését és az abból levonható következtetéseket foglalja magában.
%
%A 7. fejezetben egy áttekintést nyújtunk a dolgozatról. Részletezzük a főbb eredményeket és javaslatokat nyújtunk a jövőbeli fejlesztésekre es kutatásokra.