%----------------------------------------------------------------------------
\chapter{\bevezetes}
%----------------------------------------------------------------------------

Számos adatbázis-kezelő rendszer van a világon. A teljesítmény kulcsfontosságú jelentőséggel bír, főleg, amikor nagyméretű adatbázisokról van szó. Emiatt felmerül a kérdés, hogy sorba lehet-e őket rendezni valamilyen objektív mérce alapján. A problémák sokrétűsége azt mutatja, hogy a felhasználási terület általában nagymértékben befolyásolja az eredményt. Emellett az adatbázis-kezelő rendszereket sokféleképp kategóriába is lehet sorolni, például a matematikai modell vagy a lekérdezés nyelve alapján. Az egy kategóriába eső rendszereket általában könnyebb összehasonlítani, mint olyanokat, amik külön kategóriába esnek. Ezek a problémák szinte lehetetlen feladattá tesznek egy általános összehasonlítást. Észszerűen ezért a legtöbb teljesítménymérés egy bizonyos fajta problémára összpontosul. Egy ilyen feladat lehet a modell-validálás, melynek célja, hogy adott modellen (melyet az adatbázisban tárolunk) megkövetelt kényszereket ellenőrizzen. Ez rendkívül fontos egy fejlesztési folyamat során, mivel visszajelzést ad a programozónak és/vagy a felhasználónak a modell helyességéről. A Train Benchmark célja egy ilyen folyamat szimulálása több adatbázisrendszeren és a teljesítményük mérése, annak kiértékelése.

A Train Benchmark több adatbázis-kezelő rendszerre már meg lett valósítva. Jelen dolgozat célja egy modern adatbázis-kezelő tesztelésének implementálása, a Train Benchmark két adatbázisrendszer-tesztjének futtatása Windows operációs rendszeren és a kapott eredmények összehasonlítása.

\section{A dolgozat felépítése}

Az elkészült dolgozat a következő részekből épül fel:

A 2. fejezetben a Train Benchmark modelljének a működésének elvét mutatnánk be. Kezdve a Train Benchmarkban használt modell elemeivel, közöttük lévő kapcsolattal, majd a megfogalmazott kényszerek részletezésével, aztán a fejlesztési folyamat forgatókönyve, végül pedig azon környezetek (Jena, RDF4J) rövid bemutatása következik, melyeknek tesztjét lefuttattuk.

A 3. fejezet a gráfadatbázisokról szól. Egy egyszerű példán keresztül megmutatjuk, hogy adott esetben miért lehet gyorsabb és jobb a relációs adatbázisrendszereknél.

A 4. fejezet a Graph Engine nevű gráfadatbázis-rendszer felépítését és működését taglalja. A 3. fejezetben található példán keresztül mutatjuk be, hogy milyen módon kell használni.

Az 5. fejezetben az általunk elkészített tesztelőprogram implementációját részletezzük. Rövid forráskódokat kiragadva megpróbáljuk általánosan bemutatni, hogy hogyan épül fel. Kiegészítve a Train Benchmarkhoz már elkészült tesztek Windows operációs rendszeren való futtatási problémáinak megoldásával, illetve a fejlesztés során felmerülő nehézségekre való válaszok okainak indoklásával.

%TODO: utolsó két fejezethez tartozó bevezető

A 6. fejezet a tesztek futtatása után kapott eredmények kiértékelését foglalja magában.

A 7. fejezet összefoglalja a dolgozat során elvégzett munkát